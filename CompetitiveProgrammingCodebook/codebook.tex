\documentclass[a4paper,10pt,twocolumn,oneside]{article}
\setlength{\columnsep}{10pt}                                                              %兩欄模式的間距
\setlength{\columnseprule}{0pt}                                                                %兩欄模式間格線粗細

\usepackage{amsthm}								%定義,例題
\usepackage{amssymb}
%\usepackage[margin=2cm]{geometry}
\usepackage{fontspec}								%設定字體
\usepackage{color}
\usepackage[x11names]{xcolor}
\usepackage{listings}								%顯示code用的
%\usepackage[Glenn]{fncychap}						%排版,頁面模板
\usepackage{fancyhdr}								%設定頁首頁尾
\usepackage{graphicx}								%Graphic
\usepackage{enumerate}
\usepackage{titlesec}
\usepackage{amsmath}
\usepackage[CheckSingle, CJKmath]{xeCJK}
% \usepackage{CJKulem}
%\usepackage[T1]{fontenc}
\titlespacing{\section}{0cm}{0cm}{0cm}
\titlespacing{\subsection}{0cm}{0cm}{0cm}
\usepackage{amsmath, courier, listings, fancyhdr, graphicx}
\topmargin=0pt
\headsep=5pt
\textheight=780pt
\footskip=0pt
\voffset=-40pt
\textwidth=545pt
\marginparsep=0pt
\marginparwidth=0pt
\marginparpush=0pt
\oddsidemargin=0pt
\evensidemargin=0pt
\hoffset=-42pt

%\renewcommand\listfigurename{圖目錄}
%\renewcommand\listtablename{表目錄} 

%%%%%%%%%%%%%%%%%%%%%%%%%%%%%

\setmainfont [				%主要字型
    Path = .fonts/ttf/,
    UprightFont = *-Regular,
    BoldFont = *-Bold,
    ItalicFont = *-Italic
  ] {Consolas}

\setmonofont [        
    Path = .fonts/ttf/,
    UprightFont = *-Regular
  ] {Monaco}

% \setmainfont{Consolas}				%主要字型
% \setmonofont{Monaco}				%主要字型
% \setCJKmainfont{Noto Sans CJK TC}

% \setCJKmainfont{Consolas}			%中文字型
%\setmainfont{sourcecodepro}
\XeTeXlinebreaklocale "zh"						%中文自動換行
\XeTeXlinebreakskip = 0pt plus 1pt				%設定段落之間的距離
\setcounter{secnumdepth}{3}						%目錄顯示第三層

%%%%%%%%%%%%%%%%%%%%%%%%%%%%%
\makeatletter
\lst@CCPutMacro\lst@ProcessOther {"2D}{\lst@ttfamily{-{}}{-{}}}
\@empty\z@\@empty
\makeatother
\lstset{											% Code顯示
language=C++,										% the language of the code
basicstyle=\footnotesize\ttfamily, 						% the size of the fonts that are used for the code
%numbers=left,										% where to put the line-numbers
numberstyle=\footnotesize,						% the size of the fonts that are used for the line-numbers
stepnumber=1,										% the step between two line-numbers. If it's 1, each line  will be numbered
numbersep=5pt,										% how far the line-numbers are from the code
backgroundcolor=\color{white},					% choose the background color. You must add \usepackage{color}
showspaces=false,									% show spaces adding particular underscores
showstringspaces=false,							% underline spaces within strings
showtabs=false,									% show tabs within strings adding particular underscores
frame=false,											% adds a frame around the code
tabsize=2,											% sets default tabsize to 2 spaces
captionpos=b,										% sets the caption-position to bottom
breaklines=true,									% sets automatic line breaking
breakatwhitespace=false,							% sets if automatic breaks should only happen at whitespace
escapeinside={\%*}{*)},							% if you want to add a comment within your code
morekeywords={*},									% if you want to add more keywords to the set
keywordstyle=\bfseries\color{Blue1},
commentstyle=\itshape\color{Red4},
stringstyle=\itshape\color{Green4},
}

%%%%%%%%%%%%%%%%%%%%%%%%%%%%%

\begin{document}
\pagestyle{fancy}
\fancyfoot{}
% \fancyfoot[R]{\includegraphics[width=20pt]{ironwood.jpg}}
\fancyhead[C]{National Central University}
\fancyhead[L]{NeverCareyoU}
\fancyhead[R]{(October 15, 2022) \thepage}
\renewcommand{\headrulewidth}{0.4pt}
\renewcommand{\contentsname}{Contents} 

\scriptsize
\tableofcontents
%%%%%%%%%%%%%%%%%%%%%%%%%%%%%

% \newpage

\lstinputlisting{basic/meow}

% \lstinputlisting{others/cat.cpp}

\section{Setup}

\subsection{設定}
\lstinputlisting{basic/setup}

\subsection{Default}
\lstinputlisting{basic/cpp_lang.cpp}

% \subsection{vim 指令}
% \lstinputlisting{basic/vim}

\section{Basic}

\subsection{Binary Search}
\lstinputlisting{basic/binary.cpp}

\subsection{merge sort}
\lstinputlisting{basic/mergesort.cpp}

\subsection{四捨五入}
\lstinputlisting{basic/1.cpp}

\subsection{排序}
\lstinputlisting{basic/2.cpp}

\subsection{nth element}
\lstinputlisting{basic/nth_element.cpp}

\section{flow}

\subsection{dinic}
\lstinputlisting{flow/dinic_BCW.cpp}

\subsection{MinCostFlow}
\lstinputlisting{flow/MinCostFlow.cpp}

\subsection{Kuhn Munkres 最大完美二分匹配}
\lstinputlisting{flow/KM2.cpp}

\section{Math}

\subsection{Binary exponentiation}
\lstinputlisting{math/binary_ex.cpp}

\subsection{Euclidean gcd}
\lstinputlisting{math/gcd.cpp}

\subsection{lcm 最小公倍數}
\lstinputlisting{math/lcm.cpp}

\subsection{Miller Rabin}
\lstinputlisting{math/Miller_Rabin.cpp}

\subsection{快速乘}
\lstinputlisting{math/O(1)mul.cpp}

% \subsection{ax+by=gcd}
% \lstinputlisting{math/ax+by=gcd.cpp}

% \subsection{Result}
% \begin{itemize}
\item Lucas’ Theorem :\\
  For $n, m \in \mathbb{Z}^{*}$ and prime $P$,
  $C(m,n) \mod P$
  %= C(\frac{m}{M},n/M) * C(m\%M,n\%M) mod P
	$= \Pi ( C(m_i,n_i) )$
  where $m_i$ is the $i$-th digit of $m$ in base $P$.
\item Stirling approximation : \\
  $n!\approx\sqrt{ 2 \pi n}(\frac{n}{e})^{n}e^\frac{1}{12n}$
\item Stirling Numbers(permutation $|P|=n$ with $k$ cycles): \\
  $S(n,k) = \text{coefficient of }x^k \text{ in } \Pi_{i=0}^{n-1} (x+i)$
\item Stirling Numbers(Partition $n$ elements into $k$ non-empty set): \\
  $S(n,k) = \frac{1}{k!} \sum\limits_{j=0}^k (-1)^{k-j} {k \choose j} j^n$
\item Pick’s Theorem : $A = i + b/2 - 1$\\
  其面積$A$和內部格點數目$i$、邊上格點數目$b$的關係
\item Catalan number : $C_n = {2n \choose n}/(n+1)$\\
  $C^{n+m}_{n}-C^{n+m}_{n+1} = (m+n)! \frac{n-m+1}{n+1}\quad for \quad  n \ge m$\\
  $C_n = \frac{1}{n+1}{2n \choose n} = \frac{(2n)!}{(n+1)!n!}$\\
  $C_0 = 1 \quad  and \quad C_{n+1}= 2(\frac{2n+1}{n+2})C_n$\\
  $C_0 = 1 \quad  and \quad C_{n+1} = \sum_{i=0}^{n} C_iC_{n-i} \quad for \quad  n \ge 0$
\item Euler Characteristic: \\
  planar graph: $V-E+F-C=1$ \\
  convex polyhedron: $V-E+F=2$ \\
  $V,E,F,C$: number of vertices, edges, faces(regions), and components
\item Kirchhoff's theorem : \\
  $A_{ii} = deg(i), A_{ij} = (i,j) \in E\ ? -1 : 0$,
  Deleting any one row, one column, and cal the det(A)
\item Polya' theorem (c為方法數,m為總數): \\
  $(\sum_{i=1}^{m}{c^{gcd(i,m)}})/m$
\item Burnside lemma: \\
  $|X/G| = \frac{1}{|G|}\sum\limits_{g\in G} |X^g|$
\item 錯排公式 :  ($n$個人中,每個人皆不再原來位置的組合數): \\
  $dp[0]=1;dp[1]=0;$\\
  $dp[i]=(i-1)*(dp[i-1]+dp[i-2])$;
\item Bell數 (有$n$個人,把他們拆組的方法總數) : \\
  $B_0= 1$\\
  $B_n= \sum_{k=0}^{n} s(n,k)\quad (second-stirling)\\
  B_{n+1}= \sum_{k=0}^{n}{n \choose k} B_k$
\item Wilson's theorem :\\
  $(p-1)! \equiv -1 (mod \ p)$
\item Fermat's little theorem :\\
  $a^p \equiv a (mod \ p)$
\item Euler's totient function:\\
  $ A ^ {B ^ C} mod \ p = pow(A,pow(B,C,p-1)) mod \ p$
\item 歐拉函數降冪公式:\\
  $A^B \mod C=A^{B \mod \phi(c) + \phi(c)}\mod C$
\item 6的倍數: \\
 $(a-1)^3 + (a+1)^3 + (-a)^3 + (-a)^3 = 6a$
\end{itemize}



\section{Geometry}

\subsection{ConvexHull}
\lstinputlisting{geometry/convexhull.cpp}

\section{Graph}

\subsection{Strongly Connected Component}
\lstinputlisting{graph/kosaraju.cpp}

\subsection{BCC based on vertex}
\lstinputlisting{graph/bcc_vertex.cpp}

\subsection{graph}
\lstinputlisting{graph/graphh}

\subsection{bfs}
\lstinputlisting{graph/bfs.cpp}

\subsection{dfs}
\lstinputlisting{graph/dfs.cpp}

\subsection{dijkstra(單源最短路徑)(堆優化O(mlogn))}
\lstinputlisting{graph/dijkstra.cpp}

\subsection{bellman ford}
\lstinputlisting{graph/bellmanford.cpp}

\subsection{SPFA求最短路徑}
\lstinputlisting{graph/spfa.cpp}

\subsection{SPFA判斷負環}
\lstinputlisting{graph/spfa1.cpp}

\subsection{floyd}
\lstinputlisting{graph/floyd.cpp}

\subsection{無向圖中字典序最小歐拉路徑}
\lstinputlisting{graph/eula.cpp}

\subsection{topological sort}
\lstinputlisting{graph/topo.cpp}

% \subsection{K-th Shortest Path}
% \lstinputlisting{graph/KSP.cpp}

\section{String}

\subsection{KMP}
\lstinputlisting{string/KMP.cpp}

\subsection{Z Value}
\lstinputlisting{string/zvalue.cpp}

\section{Data Structure}

\subsection{DSU}
\lstinputlisting{dataStructure/dsu.cpp}

\subsection{trie}
\lstinputlisting{dataStructure/trie.cpp}

\subsection{Segment tree}
\lstinputlisting{dataStructure/seg_tree.cpp}

% \subsection{Treap}
% \lstinputlisting{dataStructure/treap_split_key&kth.cpp}

\section{Others}

\subsection{dp 背包問題}
\lstinputlisting{others/dp_backpack.cpp}

\subsection{dp 完全背包問題}
\lstinputlisting{others/dp_backpack1.cpp}

\subsection{dp 多重背包問題}
\lstinputlisting{others/dp_backpack2.cpp}

\subsection{LIS 最長上升子序列}
\lstinputlisting{others/lis.cpp}

\subsection{LCS 最長公共子序列}
\lstinputlisting{others/lcs.cpp}

\subsection{LCIS 最長公共上升子序列}
\lstinputlisting{others/lcis.cpp}

\end{document}